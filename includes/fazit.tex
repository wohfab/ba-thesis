\chapter{Fazit und Ausblick}
\label{chap:Fazit}

% Ergebnisse zusammenfassen und bewerten - Beantworten der Fragen aus der Einleitung - Ausblick / Offene Fragen / Angrenzende Themengebiete
% KEINE neuen Erkenntnisse / Thesen

Die Ergebnisse aus Kapitel \ref{sec:Ergebnisse} zeigen, dass nicht für alle Beiträge eine Kategorisierung möglich ist, wie sie Schmidt für die Social Web Kommunikation postuliert. 10 \% der analysierten Social Web Beiträge der Universität aus dem Juli 2018 haben sich nicht in dem Modell widerspiegeln lassen.

In der Annahme, dass das Modell der Management-Leistungen nach Jan-Hinrik Schmidt aus Abschnitt \ref{sec:jhsforschung} eine Kategorisierung aller veröffentlichter Beiträge im Social Web garantiert, ist eine vollständige Übernahme des Modells von der persönlichen auf die Institutions-Kommunikation im Social Web daher nicht möglich. 

Die kritische Begutachtung der Arbeit in Kapitel \ref{sec:Diskussion} zeigt, dass eine weniger kommunikationswissenschaftliche -- sondern im Kern computerlinguistische -- Arbeit zu repräsentativeren Ergebnissen kommen könnte. Eine solche technischer ausgelegte Arbeit könnte dann den Rahmen des Datensatzes erweitern oder gar aufheben. Sie könnte Social Web Beiträge dem Inhalt nach automatisiert analysieren. Die Beschaffung der Daten kann vollständig automatisiert über die APIs (Application Programming Interfaces -- Programmierschnittstellen) der jeweiligen Social Web Plattformen erfolgen. Für die inhaltliche Verarbeitung kommen diverse Techniken des NLP (Natural Language Processing) zum Einsatz. Eine simple Technik wäre das sogenannte TF/IDF (Term Frequency/ Inverse Document Frequency), welches durch die Berechnung der Häufigkeiten von im Text vorkommenden Begriffen, in der Lage ist, Themenbereiche eines Textes herauszufinden und so Social Web Beiträge automatisiert verschlagworten könnte (vgl. \cite{ramos2003tfidf}). Ramos selber bedient sich in seiner Arbeit weiteren computerlinguistischen Techniken, die sich unter anderem in \cite{kiss2006unsupervised}, \cite{mitkov2012coreference} und \cite{uryupina2006coreference} wiederfinden. Diese Techniken des NLP finden unter anderem Anwendung in der automatischen Textzusammenfassung. Das Zwischenziel der automatisierten Textzusammenfassung ist es, den Inhalt eines Textes programmatisch zu erfassen. Dies kann dazu genutzt werden, die Social Web Beiträge in die Kategorien des Modells von Jan-Hinrik Schmidt einzuteilen.

Gelingt eine solche Vollautomatisierung des Prozesses, können schier endlose Datenmengen in kürzester Zeit analysiert, und daraus signifikant relevante Ergebnisse abgeleitet werden, wie es \textit{per Hand} nicht machbar wäre.
