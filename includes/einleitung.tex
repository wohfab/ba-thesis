\chapter{Einleitung}
\label{chap:Einleitung}

% Zielsetzung - Problemstellung - Eingrenzung/Abgrenzung des Themas (begründet) - Aufbau - Roter Faden

Diese Abschlussarbeit widmet sich dem immer relevanter werdenden Thema der Social Media Kommunikation. Unter anderem durch Jan-Hinrik Schmidt, gibt es im deutschsprachigen Raum ein sehr aktives Forschungsfeld um die Thematiken des Social Web. Viele der Forschungsansätze untersuchen jedoch in erster Linie die Social Media Kommunikation im Rahmen der Nutzung durch Einzelpersonen.

Mit der steigenden Relevanz der sozialen Medien auch für Firmen und Institutionen, für die inzwischen das Social Media Marketing auch nicht mehr wegzudenken ist, beschäftigen sich jedoch im Verhältnis eher wenige Forschungsarbeiten.

In der heutigen Zeit, und zum laufenden Beginn der Digitalisierung, ist ein Blick auf gebräuchliche Kommunikationsmodelle angebracht, die sich, wie der Großteil der Forschung, der Kommunikation unter Personen widmet, um festzustellen, ob diese auch Anwendbarkeit finden, in der Kommunikation zwischen Firmen, Institutionen und anderen Einrichtungen, und deren Social Media Followern.

Im Speziellen möchte ich mir das Kommunikationsmodell nach Jan-Hinrik Schmidt anschauen, welches, mit den Begriffen \textit{Identitätsmanagement}, \textit{Beziehungsmanagement} und \textit{Informationsmanagement}, die Social Media Kommunikation von Einzelpersonen zu klassifizieren versucht. Dieses Modell soll im Laufe dieser Arbeit auf die Anwendbarkeit auf die Social Media Kommunikation der Universität Bielefeld -- als Beispiel einer Institution -- untersucht werden. \smallskip

Anfangen werde ich in Kapitel \ref{chap:Theorie} mit dem theoretischen Hintergrund, einer kurzen geschichtlichen Einleitung, und der Beschreibung der meist genutzten Social Media Plattformen -- sowohl allgemein als auch innerhalb der Social Media Kommunikation der Universität Bielefeld. Danach werde ich den aktuellen Stand der Forschung in Kapitel \ref{sec:aktuelleforschung} skizzieren, und mit Hilfe dessen, über begründete Forschungsfragen, zu den Hypothesen dieser Arbeit in Kapitel \ref{sec:hypothesen} überleiten.

Anschließend folgt in Kapitel \ref{chap:Empirie} die empirische Verarbeitung. Dort werde ich die genutzten Daten beschreiben und sie auf der Forschungsgrundlage aus Kapitel \ref{sec:jhsforschung} verarbeiten. Die Ergebnisse der Verarbeitung stelle ich dann in Kapitel \ref{sec:Ergebnisse} vor, und werde sie anschließend in Kapitel \ref{sec:Diskussion} analysieren und diskutieren.

Den Schluss bilden in Kapitel \ref{chap:Fazit} das Fazit meiner Analysen, die kritische Betrachtung der Herangehensweise dieser Arbeit, und ein Ausblick auf weiter zu untersuchende Forschungsfragen.

% Ziel der Arbeit ist es, das untersuchte Kommunikationsmodell zu erweitern oder zu verändern, um sich dem entwickelten Anspruch anzupassen, und die Social Media Kommunikation von Institutionen in dem Modell unterbringen zu können.
